\documentclass[journal,12pt,twocolumn]{IEEEtran}

\usepackage{setspace}
\usepackage{gensymb}

\singlespacing


\usepackage[cmex10]{amsmath}

\usepackage{amsthm}

\usepackage{mathrsfs}
\usepackage{txfonts}
\usepackage{stfloats}
\usepackage{bm}
\usepackage{cite}
\usepackage{cases}
\usepackage{subfig}

\usepackage{longtable}
\usepackage{multirow}

\usepackage{enumitem}
\usepackage{mathtools}
\usepackage{steinmetz}
\usepackage{tikz}
\usepackage{circuitikz}
\usepackage{verbatim}
\usepackage{tfrupee}
\usepackage[breaklinks=true]{hyperref}
\newcommand\myemptypage{
	\null
	\thispagestyle{empty}
	\addtocounter{page}{-1}
	\newpage
}
\usepackage{tkz-euclide}

\usetikzlibrary{calc,math}
\usepackage{listings}
    \usepackage{color}                                            %%
    \usepackage{array}                                            %%
    \usepackage{longtable}                                        %%
    \usepackage{calc}                                             %%
    \usepackage{multirow}                                         %%
    \usepackage{hhline}                                           %%
    \usepackage{ifthen}                                           %%
    \usepackage{lscape}     
\usepackage{multicol}
\usepackage{chngcntr}

\DeclareMathOperator*{\Res}{Res}

\renewcommand\thesection{\arabic{section}}
\renewcommand\thesubsection{\thesection.\arabic{subsection}}
\renewcommand\thesubsubsection{\thesubsection.\arabic{subsubsection}}

\renewcommand\thesectiondis{\arabic{section}}
\renewcommand\thesubsectiondis{\thesectiondis.\arabic{subsection}}
\renewcommand\thesubsubsectiondis{\thesubsectiondis.\arabic{subsubsection}}


\hyphenation{op-tical net-works semi-conduc-tor}
\def\inputGnumericTable{}                                 %%

\lstset{
%language=C,
frame=single, 
breaklines=true,
columns=fullflexible
}
\begin{document}


\newtheorem{theorem}{Theorem}[section]
\newtheorem{problem}{Problem}
\newtheorem{proposition}{Proposition}[section]
\newtheorem{lemma}{Lemma}[section]
\newtheorem{corollary}[theorem]{Corollary}
\newtheorem{example}{Example}[section]
\newtheorem{definition}[problem]{Definition}

\newcommand{\BEQA}{\begin{eqnarray}}
\newcommand{\EEQA}{\end{eqnarray}}
\newcommand{\define}{\stackrel{\triangle}{=}}
\bibliographystyle{IEEEtran}
\providecommand{\mbf}{\mathbf}
\providecommand{\pr}[1]{\ensuremath{\Pr\left(#1\right)}}
\providecommand{\qfunc}[1]{\ensuremath{Q\left(#1\right)}}
\providecommand{\sbrak}[1]{\ensuremath{{}\left[#1\right]}}
\providecommand{\lsbrak}[1]{\ensuremath{{}\left[#1\right.}}
\providecommand{\rsbrak}[1]{\ensuremath{{}\left.#1\right]}}
\providecommand{\brak}[1]{\ensuremath{\left(#1\right)}}
\providecommand{\lbrak}[1]{\ensuremath{\left(#1\right.}}
\providecommand{\rbrak}[1]{\ensuremath{\left.#1\right)}}
\providecommand{\cbrak}[1]{\ensuremath{\left\{#1\right\}}}
\providecommand{\lcbrak}[1]{\ensuremath{\left\{#1\right.}}
\providecommand{\rcbrak}[1]{\ensuremath{\left.#1\right\}}}
\theoremstyle{remark}
\newtheorem{rem}{Remark}
\newcommand{\sgn}{\mathop{\mathrm{sgn}}}
\providecommand{\abs}[1]{\left\vert#1\right\vert}
\providecommand{\res}[1]{\Res\displaylimits_{#1}} 
\providecommand{\norm}[1]{\left\lVert#1\right\rVert}
%\providecommand{\norm}[1]{\lVert#1\rVert}
\providecommand{\mtx}[1]{\mathbf{#1}}
\providecommand{\mean}[1]{E\left[ #1 \right]}
\providecommand{\fourier}{\overset{\mathcal{F}}{ \rightleftharpoons}}
%\providecommand{\hilbert}{\overset{\mathcal{H}}{ \rightleftharpoons}}
\providecommand{\system}{\overset{\mathcal{H}}{ \longleftrightarrow}}
	%\newcommand{\solution}[2]{\textbf{Solution:}{#1}}
\newcommand{\solution}{\noindent \textbf{Solution: }}
\newcommand{\cosec}{\,\text{cosec}\,}
\providecommand{\dec}[2]{\ensuremath{\overset{#1}{\underset{#2}{\gtrless}}}}
\newcommand{\myvec}[1]{\ensuremath{\begin{pmatrix}#1\end{pmatrix}}}
\newcommand{\mydet}[1]{\ensuremath{\begin{vmatrix}#1\end{vmatrix}}}
\numberwithin{equation}{subsection}
\makeatletter
\@addtoreset{figure}{problem}
\makeatother
\let\StandardTheFigure\thefigure
\let\vec\mathbf
\renewcommand{\thefigure}{\theproblem}
\def\putbox#1#2#3{\makebox[0in][l]{\makebox[#1][l]{}\raisebox{\baselineskip}[0in][0in]{\raisebox{#2}[0in][0in]{#3}}}}
     \def\rightbox#1{\makebox[0in][r]{#1}}
     \def\centbox#1{\makebox[0in]{#1}}
     \def\topbox#1{\raisebox{-\baselineskip}[0in][0in]{#1}}
     \def\midbox#1{\raisebox{-0.5\baselineskip}[0in][0in]{#1}}
\vspace{3cm}
\title{EE5609: Matrix Theory\\
          Assignment-13\\}
\author{M Pavan Manesh\\
EE20MTECH14017 }
\maketitle
\newpage
\bigskip
\renewcommand{\thefigure}{\theenumi}
\renewcommand{\thetable}{\theenumi}
Download all latex-tikz codes from 
%
\begin{lstlisting}
https://github.com/pavanmanesh/EE5609/tree/master/Assignment13
\end{lstlisting}
%
\section{Problem}
Let $\vec{M} =\{\vec{A}=\myvec{a&b\\c&d} : a,b,c,d \in \mathbb{Z} $ and eigen values of $\vec{A}$ $\in$ $\mathbb{Q}$$\}$ \label{main}
\begin{enumerate}
    \item $\vec{M}$ is empty
    \item $\vec{M} =\{\myvec{a&b\\c&d} : a,b,c,d \in \mathbb{Z} \}$
    \item If $\vec{A}$ $\in$ $\vec{M}$ then the eigen values of $\vec{A}$ $\in$ $\mathbb{Z}$
    \item If $\vec{A}$,$\vec{B}$ $\in$ $\vec{M}$ such that $\vec{A} \vec{B}$=$\vec{I}$ then $\mydet{\vec{A}}$ $\in$ \{+1,-1\}
\end{enumerate}
\section{\textbf{Solution}}
\pagebreak
\myemptypage
\begin{longtable}{|l|l|}
	\hline
	\multirow{3}{*}{$\vec{M}$ is empty} & \\
	& Consider $\vec{A}$=$\vec{I}$=\myvec{1&0\\0&1}.The elements of $\vec{A}$ $\in$ $\mathbb{Z}$ and it's eigen values 1 $\in$ $\mathbb{Q}$.\\
	&So, $\vec{M}$ is not empty. \\
	& \\
	\hline
	\multirow{3}{*}{$\vec{M} =\{\myvec{a&b\\c&d} : a,b,c,d \in \mathbb{Z} \}$} & \\
	& Let $\vec{A}$=\myvec{0&-1\\1&0} where elements of $\vec{A}$ $\in$ $\mathbb{Z}$.The characteristic equation can be \\ 
	& written as :\\
	& \qquad \qquad \qquad$ \lambda^2+1 = 0 \implies \lambda = \pm i$ \\
	& We see that $\lambda$ $\in$ $\mathbb{C}$ which is contradicting the main definition of $\vec{M}$ .So,this \\
	& option is not correct. \\
	& \\
	\hline
	\multirow{3}{*}{Eigen values of $\vec{A}$ $\in$ $\mathbb{Z}$} & \\
	& Given ,$\vec{A}$ $\in$ $\vec{M}$.Let $\lambda_1$,$\lambda_2$ be the eigen values of $\vec{A}$.The characteristic polynomial \\ 
	&can be written as:\\
	& \qquad \qquad \qquad$\lambda^2- tr\brak{\vec{A}}\lamda+\det{\vec{A}}=0$,where $tr\brak{\vec{A}}=\lambda_1+\lambda_2,\det{\vec{A}}=\lambda_1\lambda_2$\\ 
	& Given the eigen values $\lambda\_1$,$\lambda\_2$  $\in$ $\mathbb{Q}$,For this to be possible the discriminant \\
	& of above equation should $\in$ $\mathbb{Z}$\\
	& \qquad \qquad \qquad$\sqrt{(\lambda_1+\lambda_2)^2-4\lambda_1\lambda_2} \in \mathbb{Z}$ \\
	& \qquad \qquad$\implies \sqrt{(\lambda_1-\lambda_2)^2} \in \mathbb{Z}$ \\
	& \qquad \qquad$\implies \lambda_1-\lambda_2 \in \mathbb{Z}$ 
	This is possible when both $\lambda_1$,$\lambda_2$ $\in$ $\mathbb{Z}$.\\
	& \\
	\hline
	\multirow{3}{*}{If $\vec{A} \vec{B}$=$\vec{I}$ then $\mydet{\vec{A}}$ $\in$ \{+1,-1\}} 
	& \\
	& As $\vec{A}$,$\vec{B}$ $\in$ $\vec{M}$,$\implies$ $\mydet{\vec{A}}$,$\mydet{\vec{B}}$ $\in$ $\mathbb{Z}$ \\
	& Given $\vec{A}\vec{B}$=$\vec{I}$ $\implies$ $\mydet{\vec{A}}\mydet{\vec{B}}$=1\\
	& This is possible only when $\mydet{\vec{A}}$=$\mydet{\vec{B}}$= $\pm$ 1\\
	& \\
	\hline
	\multirow{3}{*}{Conclusion} & \\
	& options 3) and 4) are correct.\\
    & \\
    \hline
\end{longtable}
\end{document}
