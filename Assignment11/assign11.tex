\documentclass[journal,12pt,twocolumn]{IEEEtran}

\usepackage{setspace}
\usepackage{gensymb}

\singlespacing


\usepackage[cmex10]{amsmath}

\usepackage{amsthm}

\usepackage{mathrsfs}
\usepackage{txfonts}
\usepackage{stfloats}
\usepackage{bm}
\usepackage{cite}
\usepackage{cases}
\usepackage{subfig}

\usepackage{longtable}
\usepackage{multirow}

\usepackage{enumitem}
\usepackage{mathtools}
\usepackage{steinmetz}
\usepackage{tikz}
\usepackage{circuitikz}
\usepackage{verbatim}
\usepackage{tfrupee}
\usepackage[breaklinks=true]{hyperref}

\usepackage{tkz-euclide}

\usetikzlibrary{calc,math}
\usepackage{listings}
    \usepackage{color}                                            %%
    \usepackage{array}                                            %%
    \usepackage{longtable}                                        %%
    \usepackage{calc}                                             %%
    \usepackage{multirow}                                         %%
    \usepackage{hhline}                                           %%
    \usepackage{ifthen}                                           %%
    \usepackage{lscape}     
\usepackage{multicol}
\usepackage{chngcntr}

\DeclareMathOperator*{\Res}{Res}

\renewcommand\thesection{\arabic{section}}
\renewcommand\thesubsection{\thesection.\arabic{subsection}}
\renewcommand\thesubsubsection{\thesubsection.\arabic{subsubsection}}

\renewcommand\thesectiondis{\arabic{section}}
\renewcommand\thesubsectiondis{\thesectiondis.\arabic{subsection}}
\renewcommand\thesubsubsectiondis{\thesubsectiondis.\arabic{subsubsection}}


\hyphenation{op-tical net-works semi-conduc-tor}
\def\inputGnumericTable{}                                 %%

\lstset{
%language=C,
frame=single, 
breaklines=true,
columns=fullflexible
}
\begin{document}


\newtheorem{theorem}{Theorem}[section]
\newtheorem{problem}{Problem}
\newtheorem{proposition}{Proposition}[section]
\newtheorem{lemma}{Lemma}[section]
\newtheorem{corollary}[theorem]{Corollary}
\newtheorem{example}{Example}[section]
\newtheorem{definition}[problem]{Definition}

\newcommand{\BEQA}{\begin{eqnarray}}
\newcommand{\EEQA}{\end{eqnarray}}
\newcommand{\define}{\stackrel{\triangle}{=}}
\bibliographystyle{IEEEtran}
\providecommand{\mbf}{\mathbf}
\providecommand{\pr}[1]{\ensuremath{\Pr\left(#1\right)}}
\providecommand{\qfunc}[1]{\ensuremath{Q\left(#1\right)}}
\providecommand{\sbrak}[1]{\ensuremath{{}\left[#1\right]}}
\providecommand{\lsbrak}[1]{\ensuremath{{}\left[#1\right.}}
\providecommand{\rsbrak}[1]{\ensuremath{{}\left.#1\right]}}
\providecommand{\brak}[1]{\ensuremath{\left(#1\right)}}
\providecommand{\lbrak}[1]{\ensuremath{\left(#1\right.}}
\providecommand{\rbrak}[1]{\ensuremath{\left.#1\right)}}
\providecommand{\cbrak}[1]{\ensuremath{\left\{#1\right\}}}
\providecommand{\lcbrak}[1]{\ensuremath{\left\{#1\right.}}
\providecommand{\rcbrak}[1]{\ensuremath{\left.#1\right\}}}
\theoremstyle{remark}
\newtheorem{rem}{Remark}
\newcommand{\sgn}{\mathop{\mathrm{sgn}}}
\providecommand{\abs}[1]{\left\vert#1\right\vert}
\providecommand{\res}[1]{\Res\displaylimits_{#1}} 
\providecommand{\norm}[1]{\left\lVert#1\right\rVert}
%\providecommand{\norm}[1]{\lVert#1\rVert}
\providecommand{\mtx}[1]{\mathbf{#1}}
\providecommand{\mean}[1]{E\left[ #1 \right]}
\providecommand{\fourier}{\overset{\mathcal{F}}{ \rightleftharpoons}}
%\providecommand{\hilbert}{\overset{\mathcal{H}}{ \rightleftharpoons}}
\providecommand{\system}{\overset{\mathcal{H}}{ \longleftrightarrow}}
	%\newcommand{\solution}[2]{\textbf{Solution:}{#1}}
\newcommand{\solution}{\noindent \textbf{Solution: }}
\newcommand{\cosec}{\,\text{cosec}\,}
\providecommand{\dec}[2]{\ensuremath{\overset{#1}{\underset{#2}{\gtrless}}}}
\newcommand{\myvec}[1]{\ensuremath{\begin{pmatrix}#1\end{pmatrix}}}
\newcommand{\mydet}[1]{\ensuremath{\begin{vmatrix}#1\end{vmatrix}}}
\numberwithin{equation}{subsection}
\makeatletter
\@addtoreset{figure}{problem}
\makeatother
\let\StandardTheFigure\thefigure
\let\vec\mathbf
\renewcommand{\thefigure}{\theproblem}
\def\putbox#1#2#3{\makebox[0in][l]{\makebox[#1][l]{}\raisebox{\baselineskip}[0in][0in]{\raisebox{#2}[0in][0in]{#3}}}}
     \def\rightbox#1{\makebox[0in][r]{#1}}
     \def\centbox#1{\makebox[0in]{#1}}
     \def\topbox#1{\raisebox{-\baselineskip}[0in][0in]{#1}}
     \def\midbox#1{\raisebox{-0.5\baselineskip}[0in][0in]{#1}}
\vspace{3cm}
\title{EE5609: Matrix Theory\\
          Assignment-11\\}
\author{M Pavan Manesh\\
EE20MTECH14017 }
\maketitle
\newpage
\bigskip
\renewcommand{\thefigure}{\theenumi}
\renewcommand{\thetable}{\theenumi}
\begin{abstract}
This document explains regarding the linear operator
\end{abstract}
Download all latex-tikz codes from 
%
\begin{lstlisting}
https://github.com/pavanmanesh/EE5609/tree/master/Assignment11
\end{lstlisting}
%
\section{Problem}
Let $\mathbb{V}$ be the set of complex numbers regarded as vector space over the field of real numbers.We define a function $\vec{T}$ from $\mathbb{V}$ into the space of 2x2 matrices as follows.If z=x+iy with x and y real numbers,then 
\begin{align}
    \vec{T}(z)=\myvec{x+7y&5y\\-10y&x-7y}\label{eq:1}
\end{align}
Verify that
\begin{align}
    \vec{T}(z_1z_2)=\vec{T}(z_1)\vec{T}(z_2) \label{eq:1}
\end{align}
\section{Theory}
The product of two Kronecker products yields another Kronecker product:
\begin{align}
    (\vec{A} \otimes\vec{B}) (\vec{C} \otimes\vec{D})&=
    (\vec{A} \vec{C})\otimes (\vec{B} \vec{D}) \label{eq:eq_th}
\end{align}
\section{Solution}
Given,
\begin{align}
 \vec{T}(\vec{z})=\myvec{x+7y&5y\\-10y&x-7y}\\
 \vec{x}=\myvec{x\\y}\\
 \vec{T}(\vec{x})=\myvec{\myvec{1&7}\vec{x}&\myvec{0&5}\vec{x}\\\myvec{0&-10}\vec{x}&\myvec{1&-7}\vec{x}}\end{align}\begin{align}
 =\myvec{\myvec{1&7\\0&-10}\vec{x}&\myvec{0&5\\1&-7}\vec{x}}\\
 \text{Let } \vec{A}=\myvec{1&7\\0&-10}\\
 \vec{B}=\myvec{0&5\\1&-7}\\
 \implies  \vec{T}(\vec{x})=\myvec{\vec{A}\vec{x}&\vec{B}\vec{x}}\\
  \vec{T}(\vec{x})=\myvec{\vec{A}&\vec{B}}\myvec{\vec{x}&\vec{0}\\\vec{0}&\vec{x}}\label{1}
\end{align}
The diagonal block matrix can be expressed as the kronecker product of $\vec{I}$ and $\vec{x}$
\begin{align}
    \vec{I}\otimes\vec{x}=\myvec{\vec{x}&\vec{0}\\\vec{0}&\vec{x}}
\end{align}
We can write \eqref{1} as
\begin{align}
    \vec{T} (\vec{x}) = \myvec{\vec{A} & \vec{B}} (\vec{I} \otimes \vec{x})
\end{align}
Starting with RHS of \eqref{eq:1}
\begin{align}
    \vec{T}(z_1)\vec{T}(z_2)= \myvec{\vec{A} & \vec{B}} (\vec{I} \otimes \vec{z_1})\myvec{\vec{A} & \vec{B}} (\vec{I} \otimes \vec{z_2}) \label{eq:1}
\end{align}
If
\begin{align}
    (\vec{I} \otimes \vec{z_1})\myvec{\vec{A} & \vec{B}} (\vec{I} \otimes \vec{z_2})=(\vec{I} \otimes \vec{z_1z_2})
    \intertext{then,we can write \eqref{eq:1} as }
    \vec{T}(z_1)\vec{T}(z_2)=\myvec{\vec{A} & \vec{B}}(\vec{I} \otimes \vec{z_1z_2})=\vec{T}(z_1z_2)
\end{align}
\end{document}
