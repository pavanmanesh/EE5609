\documentclass[journal,12pt]{IEEEtran}
\usepackage{longtable}
\usepackage{setspace}
\usepackage{gensymb}
\singlespacing
\usepackage[cmex10]{amsmath}
\newcommand\myemptypage{
	\null
	\thispagestyle{empty}
	\addtocounter{page}{-1}
	\newpage
}
\usepackage{amsthm}
\usepackage{mdframed}
\usepackage{mathrsfs}
\usepackage{txfonts}
\usepackage{stfloats}
\usepackage{bm}
\usepackage{cite}
\usepackage{cases}
\usepackage{subfig}

\usepackage{longtable}
\usepackage{multirow}

\usepackage{enumitem}
\usepackage{mathtools}
\usepackage{steinmetz}
\usepackage{tikz}
\usepackage{circuitikz}
\usepackage{verbatim}
\usepackage{tfrupee}
\usepackage[breaklinks=true]{hyperref}
\usepackage{graphicx}
\usepackage{tkz-euclide}

\usetikzlibrary{calc,math}
\usepackage{listings}
    \usepackage{color}                                            %%
    \usepackage{array}                                            %%
    \usepackage{longtable}                                        %%
    \usepackage{calc}                                             %%
    \usepackage{multirow}                                         %%
    \usepackage{hhline}                                           %%
    \usepackage{ifthen}                                           %%
    \usepackage{lscape}     
\usepackage{multicol}
\usepackage{chngcntr}

\DeclareMathOperator*{\Res}{Res}

\renewcommand\thesection{\arabic{section}}
\renewcommand\thesubsection{\thesection.\arabic{subsection}}
\renewcommand\thesubsubsection{\thesubsection.\arabic{subsubsection}}

\renewcommand\thesectiondis{\arabic{section}}
\renewcommand\thesubsectiondis{\thesectiondis.\arabic{subsection}}
\renewcommand\thesubsubsectiondis{\thesubsectiondis.\arabic{subsubsection}}


\hyphenation{op-tical net-works semi-conduc-tor}
\def\inputGnumericTable{}                                 %%

\lstset{
%language=C,
frame=single, 
breaklines=true,
columns=fullflexible
}
\begin{document}
\onecolumn

\newtheorem{theorem}{Theorem}[section]
\newtheorem{problem}{Problem}
\newtheorem{proposition}{Proposition}[section]
\newtheorem{lemma}{Lemma}[section]
\newtheorem{corollary}[theorem]{Corollary}
\newtheorem{example}{Example}[section]
\newtheorem{definition}[problem]{Definition}

\newcommand{\BEQA}{\begin{eqnarray}}
\newcommand{\EEQA}{\end{eqnarray}}
\newcommand{\define}{\stackrel{\triangle}{=}}
\bibliographystyle{IEEEtran}
\raggedbottom
\setlength{\parindent}{0pt}
\providecommand{\mbf}{\mathbf}
\providecommand{\pr}[1]{\ensuremath{\Pr\left(#1\right)}}
\providecommand{\qfunc}[1]{\ensuremath{Q\left(#1\right)}}
\providecommand{\sbrak}[1]{\ensuremath{{}\left[#1\right]}}
\providecommand{\lsbrak}[1]{\ensuremath{{}\left[#1\right.}}
\providecommand{\rsbrak}[1]{\ensuremath{{}\left.#1\right]}}
\providecommand{\brak}[1]{\ensuremath{\left(#1\right)}}
\providecommand{\lbrak}[1]{\ensuremath{\left(#1\right.}}
\providecommand{\rbrak}[1]{\ensuremath{\left.#1\right)}}
\providecommand{\cbrak}[1]{\ensuremath{\left\{#1\right\}}}
\providecommand{\lcbrak}[1]{\ensuremath{\left\{#1\right.}}
\providecommand{\rcbrak}[1]{\ensuremath{\left.#1\right\}}}
\theoremstyle{remark}
\newtheorem{rem}{Remark}
\newcommand{\sgn}{\mathop{\mathrm{sgn}}}
\providecommand{\abs}[1]{\left\vert#1\right\vert}
\providecommand{\res}[1]{\Res\displaylimits_{#1}} 
\providecommand{\norm}[1]{\left\lVert#1\right\rVert}
%\providecommand{\norm}[1]{\lVert#1\rVert}
\providecommand{\mtx}[1]{\mathbf{#1}}
\providecommand{\mean}[1]{E\left[ #1 \right]}
\providecommand{\fourier}{\overset{\mathcal{F}}{ \rightleftharpoons}}
%\providecommand{\hilbert}{\overset{\mathcal{H}}{ \rightleftharpoons}}
\providecommand{\system}{\overset{\mathcal{H}}{ \longleftrightarrow}}
	%\newcommand{\solution}[2]{\textbf{Solution:}{#1}}
\newcommand{\solution}{\noindent \textbf{Solution: }}
\newcommand{\cosec}{\,\text{cosec}\,}
\providecommand{\dec}[2]{\ensuremath{\overset{#1}{\underset{#2}{\gtrless}}}}
\newcommand{\myvec}[1]{\ensuremath{\begin{pmatrix}#1\end{pmatrix}}}
\newcommand{\mydet}[1]{\ensuremath{\begin{vmatrix}#1\end{vmatrix}}}
\numberwithin{equation}{subsection}
\makeatletter
\@addtoreset{figure}{problem}
\makeatother
\let\StandardTheFigure\thefigure
\let\vec\mathbf
\renewcommand{\thefigure}{\theproblem}
\def\putbox#1#2#3{\makebox[0in][l]{\makebox[#1][l]{}\raisebox{\baselineskip}[0in][0in]{\raisebox{#2}[0in][0in]{#3}}}}
     \def\rightbox#1{\makebox[0in][r]{#1}}
     \def\centbox#1{\makebox[0in]{#1}}
     \def\topbox#1{\raisebox{-\baselineskip}[0in][0in]{#1}}
     \def\midbox#1{\raisebox{-0.5\baselineskip}[0in][0in]{#1}}
\vspace{3cm}
\title{Assignment 14}
\author{M Pavan Manesh - EE20MTECH14017}
\maketitle
\bigskip
\renewcommand{\thefigure}{\theenumi}
\renewcommand{\thetable}{\theenumi}
%
Download the latex-tikz codes from 
%
\begin{lstlisting}
https://github.com/pavanmanesh/EE5609/tree/master/Assignment14
\end{lstlisting}
\section{\textbf{Problem}}
Let $\vec{T}$ be the linear operator on a $n-$ dimensional vector space $\vec{V}$ and suppose that  $\vec{T}$ has an n distinct characteristic values.Prove that $\vec{T}$ is diagonalizable.
\section{\textbf{Results used}}
\begin{longtable}{|l|l|}
	\hline
	\multirow{3}{*}{Diagonalizable} 
	& \\
	& A linear operator $\vec{T}$ on a finite-dimensional vector space $\vec{V}$ is diagonalizable if and\\
    &only if there exists an basis of $\vec{V}$, consisting of eigen vectors of $\vec{T}$ \\ 
	&\\
	\hline
	\multirow{3}{*}{Theorem}
	& \\
	& If $\vec{v}_1,\vec{v}_2,\dots,\vec{v}_k$ are eigenvectors of a linear operator $\vec{T}$ with distinct eigen\\
	& values $\lambda_1,\lambda_2,\dots,\lambda_k$,then $\vec{v}_1,\vec{v}_2,\dots,\vec{v}_k$ are linearly independent.
	\\
	&\\
	&Let $\vec{v_1}$ and $\vec{v_2}$ be the eigen vectors corresponding to eigen values $\lambda_1$ and $\lambda_2$\\
	&\quad$\implies \vec{T}(\vec{v}_1)$=$\lambda_1\vec{v}_1, \vec{T}(\vec{v}_2)$=$\lambda_2\vec{v}_2$\\	
	&\\
	&Let the linear combination of two eigen vectors be \\
	& \quad \quad\quad$a_1\vec{v}_1+a_2\vec{v}_2=0$ \qquad \qquad \dots \brak{1}\\
	& Applying $\vec{T}$ on both sides , we get\\
	& \quad \quad\quad $\vec{T}$($a_1\vec{v}_1+a_2\vec{v}_2)$=0
	\\
	& \qquad  $\implies a_1\vec{T}(\vec{v}_1)+ a_2\vec{T}(\vec{v}_2)$=0\\
	& \quad $\implies a_1\lambda_1\vec{v}_1+a_2\lambda_2\vec{v}_2=0$  \qquad \dots\brak{2}\\
	&\\
	& Multiplying \brak{1} by $\lambda_1$,we get\\
	& \quad \quad\quad$a_1\lambda_1\vec{v}_1+a_2\lambda_1\vec{v}_2=0$ \qquad \dots \brak{3}\\ 
	&\\
    & Subtracting \brak{2} and \brak{3},we get \\
    &\quad \quad\quad$a_2(\lambda_2-\lambda_1)\vec{v}_2=0$\\
    & As $\lambda_1,\lambda_2$ are distinct \\
    & \qquad$\implies \lambda_2-\lambda_1 \neq 0$\\
    & So,We can say that \\
    &\quad\quad$a_2=0$\\
    & Substituting this in $\brak{1}$, we get $a_1=0$\\
    & Therefore $\vec{v}_1,\vec{v}_2$ are linearly independent \\
    & By induction ,for n distinct eigen values $\vec{v}_1,\vec{v}_2,\dots,\vec{v}_n$ are linearly independent.\\
    & \\
    \hline
\end{longtable}
\section{\textbf{Solution}}
\begin{longtable}{|l|l|}
	\hline
	\multirow{3}{*}{Given} & \\
	& $\vec{T}$ has an n distinct characteristic values and dim$(\vec{V})$ = n\\
    & \\
    \hline
	\multirow{3}{*}{$\vec{T}$ is diagonalizable}
	& \\
	& Let $\lambda_1,\lambda_2,\dots,\lambda_n$ be distinct eigen values of $\vec{T}$ and let $\vec{v}_1,\vec{v}_2,\dots,\vec{v}_n$ be the eigen\\
	& vectors of $\vec{T}$..From above results we can state that \cbrak{\vec{v}_1,\vec{v}_2,\dots,\vec{v}_n} is linearly\\
	&independent.And also given that dim$(\vec{V})$ = n .So,this set forms a basis of $\vec{V}$.\\
	&\cbrak{\vec{v}_1,\vec{v}_2,\dots,\vec{v}_n} is a basis for $\vec{V}$ consisting of eigen vectors of $\vec{T}$.\\
    &So, $\vec{T}$ is diagonalizable.\\
	\hline
\end{longtable}
\end{document}
