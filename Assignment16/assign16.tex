\documentclass[journal,12pt]{IEEEtran}
\usepackage{longtable}
\usepackage{setspace}
\usepackage{gensymb}
\singlespacing
\usepackage[cmex10]{amsmath}
\newcommand\myemptypage{
	\null
	\thispagestyle{empty}
	\addtocounter{page}{-1}
	\newpage
}
\usepackage{amsthm}
\usepackage{mdframed}
\usepackage{mathrsfs}
\usepackage{txfonts}
\usepackage{stfloats}
\usepackage{bm}
\usepackage{cite}
\usepackage{cases}
\usepackage{subfig}

\usepackage{longtable}
\usepackage{multirow}

\usepackage{enumitem}
\usepackage{mathtools}
\usepackage{steinmetz}
\usepackage{tikz}
\usepackage{circuitikz}
\usepackage{verbatim}
\usepackage{tfrupee}
\usepackage[breaklinks=true]{hyperref}
\usepackage{graphicx}
\usepackage{tkz-euclide}

\usetikzlibrary{calc,math}
\usepackage{listings}
    \usepackage{color}                                            %%
    \usepackage{array}                                            %%
    \usepackage{longtable}                                        %%
    \usepackage{calc}                                             %%
    \usepackage{multirow}                                         %%
    \usepackage{hhline}                                           %%
    \usepackage{ifthen}                                           %%
    \usepackage{lscape}     
\usepackage{multicol}
\usepackage{chngcntr}

\DeclareMathOperator*{\Res}{Res}

\renewcommand\thesection{\arabic{section}}
\renewcommand\thesubsection{\thesection.\arabic{subsection}}
\renewcommand\thesubsubsection{\thesubsection.\arabic{subsubsection}}

\renewcommand\thesectiondis{\arabic{section}}
\renewcommand\thesubsectiondis{\thesectiondis.\arabic{subsection}}
\renewcommand\thesubsubsectiondis{\thesubsectiondis.\arabic{subsubsection}}


\hyphenation{op-tical net-works semi-conduc-tor}
\def\inputGnumericTable{}                                 %%

\lstset{
%language=C,
frame=single, 
breaklines=true,
columns=fullflexible
}
\begin{document}
\onecolumn

\newtheorem{theorem}{Theorem}[section]
\newtheorem{problem}{Problem}
\newtheorem{proposition}{Proposition}[section]
\newtheorem{lemma}{Lemma}[section]
\newtheorem{corollary}[theorem]{Corollary}
\newtheorem{example}{Example}[section]
\newtheorem{definition}[problem]{Definition}

\newcommand{\BEQA}{\begin{eqnarray}}
\newcommand{\EEQA}{\end{eqnarray}}
\newcommand{\define}{\stackrel{\triangle}{=}}
\bibliographystyle{IEEEtran}
\raggedbottom
\setlength{\parindent}{0pt}
\providecommand{\mbf}{\mathbf}
\providecommand{\pr}[1]{\ensuremath{\Pr\left(#1\right)}}
\providecommand{\qfunc}[1]{\ensuremath{Q\left(#1\right)}}
\providecommand{\sbrak}[1]{\ensuremath{{}\left[#1\right]}}
\providecommand{\lsbrak}[1]{\ensuremath{{}\left[#1\right.}}
\providecommand{\rsbrak}[1]{\ensuremath{{}\left.#1\right]}}
\providecommand{\brak}[1]{\ensuremath{\left(#1\right)}}
\providecommand{\lbrak}[1]{\ensuremath{\left(#1\right.}}
\providecommand{\rbrak}[1]{\ensuremath{\left.#1\right)}}
\providecommand{\cbrak}[1]{\ensuremath{\left\{#1\right\}}}
\providecommand{\lcbrak}[1]{\ensuremath{\left\{#1\right.}}
\providecommand{\rcbrak}[1]{\ensuremath{\left.#1\right\}}}
\theoremstyle{remark}
\newtheorem{rem}{Remark}
\newcommand{\sgn}{\mathop{\mathrm{sgn}}}
\providecommand{\abs}[1]{\left\vert#1\right\vert}
\providecommand{\res}[1]{\Res\displaylimits_{#1}} 
\providecommand{\norm}[1]{\left\lVert#1\right\rVert}
%\providecommand{\norm}[1]{\lVert#1\rVert}
\providecommand{\mtx}[1]{\mathbf{#1}}
\providecommand{\mean}[1]{E\left[ #1 \right]}
\providecommand{\fourier}{\overset{\mathcal{F}}{ \rightleftharpoons}}
%\providecommand{\hilbert}{\overset{\mathcal{H}}{ \rightleftharpoons}}
\providecommand{\system}{\overset{\mathcal{H}}{ \longleftrightarrow}}
	%\newcommand{\solution}[2]{\textbf{Solution:}{#1}}
\newcommand{\solution}{\noindent \textbf{Solution: }}
\newcommand{\cosec}{\,\text{cosec}\,}
\providecommand{\dec}[2]{\ensuremath{\overset{#1}{\underset{#2}{\gtrless}}}}
\newcommand{\myvec}[1]{\ensuremath{\begin{pmatrix}#1\end{pmatrix}}}
\newcommand{\mydet}[1]{\ensuremath{\begin{vmatrix}#1\end{vmatrix}}}
\numberwithin{equation}{subsection}
\makeatletter
\@addtoreset{figure}{problem}
\makeatother
\let\StandardTheFigure\thefigure
\let\vec\mathbf
\renewcommand{\thefigure}{\theproblem}
\def\putbox#1#2#3{\makebox[0in][l]{\makebox[#1][l]{}\raisebox{\baselineskip}[0in][0in]{\raisebox{#2}[0in][0in]{#3}}}}
     \def\rightbox#1{\makebox[0in][r]{#1}}
     \def\centbox#1{\makebox[0in]{#1}}
     \def\topbox#1{\raisebox{-\baselineskip}[0in][0in]{#1}}
     \def\midbox#1{\raisebox{-0.5\baselineskip}[0in][0in]{#1}}
\vspace{3cm}
\title{Assignment 16}
\author{M Pavan Manesh - EE20MTECH14017}
\maketitle
\bigskip
\renewcommand{\thefigure}{\theenumi}
\renewcommand{\thetable}{\theenumi}
%
Download the latex-tikz codes from 
%
\begin{lstlisting}
https://github.com/pavanmanesh/EE5609/tree/master/Assignment16
\end{lstlisting}
\section{\textbf{Problem}}
True or false? If the triangular matrix $\vec{A}$ is similar to a diagonal matrix, then $\vec{A}$ is already diagonal.
\section{\textbf{Definitions}}
\renewcommand{\thetable}{1}
\begin{longtable}{|l|l|}
	\hline
	\multirow{3}{*}{Characteristic Polynomial} 
	& \\
	& For an $n\times n$ matrix $\vec{A}$, characteristic polynomial is defined by,\\
	&\\
	& $\qquad\qquad\qquad p\brak{x}=\mydet{x\Vec{I}-\Vec{A}}$\\
	&\\
	\hline
	\multirow{3}{*}{Minimal Polynomial} 
	&\\
	& Minimal polynomial $m\brak{x}$ is the smallest factor of characteristic polynomial\\
	& $p\brak{x}$ such that,\\
	&\\
	& $\qquad \qquad \qquad m\brak{\vec{A}}=0$\\
	&\\
    \hline
    \multirow{3}{*}{Theorem} 
	&\\
	& Let $\vec{V}$ be a finite-dimensional vector space over the field $\vec{F}$ and let $\vec{T}$\\
	&be a linear operator on $\vec{V}$.Then $\vec{T}$ is diagonalizable if and only if \\
	&the minimal polynomial for $\Vec{T}$ has the form \\
	&\parbox{10cm}
	{\begin{align}
	p = \brak{x-c_1}\dots\brak{x-c_k} \label{eq:1}
	\end{align}}\\ 
	&\text{where} $c_1,c_2,...,c_k$ are distinct elements of $F$. \\
	&\\
	\hline
	\multirow{3}{*}{Diagonalizable}
	&\\
	&$\vec{A}$ is called diagonalizable if it is similar to diagnol matrix $\vec{B}$ i.e., if $\exists$ an \\
	&invertible matrix $\vec{P}$ such that\\
	&\parbox{10cm}
	{\begin{align}
    \vec{B}=\vec{P}^{-1}\vec{A}\vec{P}
	\end{align}}\\
	&\\
    \hline
    \caption{Definitions and theorem used}
    \label{table:1}
\end{longtable}
\newpage
\section{\textbf{Solution}}
\renewcommand{\thetable}{2}
\begin{longtable}{|l|l|}
	\hline
	\multirow{3}{*}{Given} & \\
	& The triangular matrix $\vec{A}$ is similar to a diagonal matrix.\\
    & \\
    \hline
	\multirow{3}{*}{Example}
	& \\
	& Let\\
	&\parbox{10cm}
	{\begin{align}
	\vec{A}=\myvec{1&2\\0&3}
	\end{align}}\\
	&We can see that $\vec{A}$ is triangular but not diagonal.\\
	&\\
	\hline
	\multirow{3}{*}{Characteristic polynomial}
	&\\
    &\parbox{10cm}
	{\begin{align}
	\mydet{x\Vec{I}-\Vec{A}}=\mydet{x-1&2\\0&x-3}\\
	=\brak{x-1}\brak{x-3}
	\end{align}}\\
	&\\
	\hline
	\multirow{3}{*}{Minimal polynomial}
	&\\
	&As the eigen values are distinct,minimal polynomial\\
    &\parbox{10cm}
	{\begin{align}
	m\brak{x}=\brak{x-1}\brak{x-3}
	\end{align}}\\
	&\\
	\hline
	\multirow{3}{*}{Diagonalizable}
	&\\
	&From theorem \eqref{eq:1},We can say that $\vec{A}$ diagonalizable i.e.,\\
	&it is similar to a diagnol matrix.\\
	&\\
	\hline
	\multirow{3}{*}{Finding matrix similar to $\vec{A}$}
	&\\
	&The eigen values are \\
	&\parbox{10cm}
	{\begin{align}
	\lambda_1=1,\lambda_2=3
	\end{align}}\\
	&The eigen vectors are\\
	&\parbox{10cm}
	{\begin{align}
	\brak{\vec{A}-\lambda_i\vec{I}}\vec{x_i}=0\\
	\lambda_1=1
	\implies \myvec{0&2\\0&2}\vec{x_1}=\myvec{0\\0}
	\implies \vec{x_1}=\myvec{1\\0}\\
	\lambda_2=3
	\implies \myvec{-2&2\\0&0}\vec{x_2}=\myvec{0\\0}
	\implies \vec{x_2}=\myvec{1\\1}
	\end{align}}\\
	& The invertible matrix\\
	&\parbox{10cm}
	{\begin{align}
	\vec{P}=\myvec{\vec{x_1} &\vec{x_2}}=\myvec{1&1\\0&1}
	\end{align}}\\
	&The diagnol matrix similar to $\vec{A}$\\
	&\parbox{10cm}
	{\begin{align}
	\vec{B}=\vec{P}^{-1}\vec{A}\vec{P}=\myvec{1&-1\\0&1}\myvec{1&2\\0&3}\myvec{1&1\\0&1}\\
	\vec{B}=\myvec{1&0\\0&3}
	\end{align}}\\
	&\\
    \hline
	\multirow{3}{*}{Conclusion} & \\
	& From above,we can say that $\vec{A}$  need not be diagonal to satisfy\\ 
	& given conditions.So, given statement is false.\\
	& \\
	\hline
	\caption{Finding minimal polynomial and similar matrix}
    \label{table:2}
\end{longtable}
\end{document}
