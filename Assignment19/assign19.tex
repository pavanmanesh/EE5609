\documentclass[journal,12pt]{IEEEtran}
\usepackage{longtable}
\usepackage{setspace}
\usepackage{gensymb}
\singlespacing
\usepackage[cmex10]{amsmath}
\newcommand\myemptypage{
	\null
	\thispagestyle{empty}
	\addtocounter{page}{-1}
	\newpage
}
\usepackage{amsthm}
\usepackage{mdframed}
\usepackage{mathrsfs}
\usepackage{txfonts}
\usepackage{stfloats}
\usepackage{bm}
\usepackage{cite}
\usepackage{cases}
\usepackage{subfig}

\usepackage{longtable}
\usepackage{multirow}

\usepackage{enumitem}
\usepackage{mathtools}
\usepackage{steinmetz}
\usepackage{tikz}
\usepackage{circuitikz}
\usepackage{verbatim}
\usepackage{tfrupee}
\usepackage[breaklinks=true]{hyperref}
\usepackage{graphicx}
\usepackage{tkz-euclide}

\usetikzlibrary{calc,math}
\usepackage{listings}
    \usepackage{color}                                            %%
    \usepackage{array}                                            %%
    \usepackage{longtable}                                        %%
    \usepackage{calc}                                             %%
    \usepackage{multirow}                                         %%
    \usepackage{hhline}                                           %%
    \usepackage{ifthen}                                           %%
    \usepackage{lscape}     
\usepackage{multicol}
\usepackage{chngcntr}

\DeclareMathOperator*{\Res}{Res}

\renewcommand\thesection{\arabic{section}}
\renewcommand\thesubsection{\thesection.\arabic{subsection}}
\renewcommand\thesubsubsection{\thesubsection.\arabic{subsubsection}}

\renewcommand\thesectiondis{\arabic{section}}
\renewcommand\thesubsectiondis{\thesectiondis.\arabic{subsection}}
\renewcommand\thesubsubsectiondis{\thesubsectiondis.\arabic{subsubsection}}


\hyphenation{op-tical net-works semi-conduc-tor}
\def\inputGnumericTable{}                                 %%

\lstset{
%language=C,
frame=single, 
breaklines=true,
columns=fullflexible
}
\begin{document}
\onecolumn

\newtheorem{theorem}{Theorem}[section]
\newtheorem{problem}{Problem}
\newtheorem{proposition}{Proposition}[section]
\newtheorem{lemma}{Lemma}[section]
\newtheorem{corollary}[theorem]{Corollary}
\newtheorem{example}{Example}[section]
\newtheorem{definition}[problem]{Definition}

\newcommand{\BEQA}{\begin{eqnarray}}
\newcommand{\EEQA}{\end{eqnarray}}
\newcommand{\define}{\stackrel{\triangle}{=}}
\bibliographystyle{IEEEtran}
\raggedbottom
\setlength{\parindent}{0pt}
\providecommand{\mbf}{\mathbf}
\providecommand{\pr}[1]{\ensuremath{\Pr\left(#1\right)}}
\providecommand{\qfunc}[1]{\ensuremath{Q\left(#1\right)}}
\providecommand{\sbrak}[1]{\ensuremath{{}\left[#1\right]}}
\providecommand{\lsbrak}[1]{\ensuremath{{}\left[#1\right.}}
\providecommand{\rsbrak}[1]{\ensuremath{{}\left.#1\right]}}
\providecommand{\brak}[1]{\ensuremath{\left(#1\right)}}
\providecommand{\lbrak}[1]{\ensuremath{\left(#1\right.}}
\providecommand{\rbrak}[1]{\ensuremath{\left.#1\right)}}
\providecommand{\cbrak}[1]{\ensuremath{\left\{#1\right\}}}
\providecommand{\lcbrak}[1]{\ensuremath{\left\{#1\right.}}
\providecommand{\rcbrak}[1]{\ensuremath{\left.#1\right\}}}
\theoremstyle{remark}
\newtheorem{rem}{Remark}
\newcommand{\sgn}{\mathop{\mathrm{sgn}}}
\providecommand{\abs}[1]{\left\vert#1\right\vert}
\providecommand{\res}[1]{\Res\displaylimits_{#1}} 
\providecommand{\norm}[1]{\left\lVert#1\right\rVert}
%\providecommand{\norm}[1]{\lVert#1\rVert}
\providecommand{\mtx}[1]{\mathbf{#1}}
\providecommand{\mean}[1]{E\left[ #1 \right]}
\providecommand{\fourier}{\overset{\mathcal{F}}{ \rightleftharpoons}}
%\providecommand{\hilbert}{\overset{\mathcal{H}}{ \rightleftharpoons}}
\providecommand{\system}{\overset{\mathcal{H}}{ \longleftrightarrow}}
	%\newcommand{\solution}[2]{\textbf{Solution:}{#1}}
\newcommand{\solution}{\noindent \textbf{Solution: }}
\newcommand{\cosec}{\,\text{cosec}\,}
\providecommand{\dec}[2]{\ensuremath{\overset{#1}{\underset{#2}{\gtrless}}}}
\newcommand{\myvec}[1]{\ensuremath{\begin{pmatrix}#1\end{pmatrix}}}
\newcommand{\mydet}[1]{\ensuremath{\begin{vmatrix}#1\end{vmatrix}}}
\numberwithin{equation}{subsection}
\makeatletter
\@addtoreset{figure}{problem}
\makeatother
\let\StandardTheFigure\thefigure
\let\vec\mathbf
\renewcommand{\thefigure}{\theproblem}
\def\putbox#1#2#3{\makebox[0in][l]{\makebox[#1][l]{}\raisebox{\baselineskip}[0in][0in]{\raisebox{#2}[0in][0in]{#3}}}}
     \def\rightbox#1{\makebox[0in][r]{#1}}
     \def\centbox#1{\makebox[0in]{#1}}
     \def\topbox#1{\raisebox{-\baselineskip}[0in][0in]{#1}}
     \def\midbox#1{\raisebox{-0.5\baselineskip}[0in][0in]{#1}}
\vspace{3cm}
\title{Assignment 19}
\author{M Pavan Manesh - EE20MTECH14017}
\maketitle
\bigskip
\renewcommand{\thefigure}{\theenumi}
\renewcommand{\thetable}{\theenumi}
%
Download the latex-tikz codes from 
%
\begin{lstlisting}
https://github.com/pavanmanesh/EE5609/tree/master/Assignment19
\end{lstlisting}
\section{\textbf{Problem}}
Let $\vec{A}$ be an invertible real $n \times n$ matrix .Define a function F: $\mathbb{R}^n\times\mathbb{R}^n\rightarrow\mathbb{R}$  by F$(\vec{x},\vec{y})=\langle \vec{A}\vec{x},\vec{y} \rangle$ where $\langle \vec{x},\vec{y} \rangle$ denotes the inner product of $\vec{x}$ and  $\vec{y}$.Let  $DF(\vec{x},\vec{y})$ denote the derivative of F at $(\vec{x},\vec{y})$ which is a linear tranformation from $\mathbb{R}^n\times\mathbb{R}^n\rightarrow\mathbb{R}$.Then
\begin{enumerate}
\item If $\vec{x} \neq 0$,then $DF(\vec{x},0 ) \neq 0$
\item If $\vec{y} \neq 0$,then $DF(0,\vec{y} ) \neq 0$
\item If $(\vec{x},\vec{y}) \neq 0$,then $DF(\vec{x},\vec{y} ) \neq 0$
\item If $\vec{x} = 0$ or $\vec{y}=0$,then $DF(\vec{x},\vec{y})= 0$
\end{enumerate}
\section{\textbf{Definitions}}
\renewcommand{\thetable}{1}
\begin{longtable}{|l|l|}
	\hline
	\multirow{3}{*}{Invertible} 
	& \\
	& A square matrix is invertible if and only if it does not have a zero eigenvalue.\\ 
	& So, from the definition of eigen vector we can write that \\
	&\parbox{10cm}
	{\begin{align}
	\vec{A}\vec{x} \neq 0 \label{eq:1}
	\end{align}}\\ 
	&The transpose of an invertible matrix is also invertible with inverse $(\vec{A}^{-1})^T$.\\
	&\parbox{10cm}
	{\begin{align}
	\vec{A}\vec{A}^{-1}=\vec{I}
	\implies(\vec{A}^{-1})^T\vec{A}^T=\vec{I}^T=\vec{I}\\
	\text{So,similarly we can say that} \nonumber \\
	\vec{A}^T\vec{y} \neq 0 \label{eq:1.1}
	\end{align}}\\ 
	\hline
	\multirow{3}{*}{Derivative of F} 
	&\\
	& Suppose F: $\mathbb{R}^n\rightarrow\mathbb{R}^m$,the derivative of a function F is given by the\\
	&Jacobian matrix\\
	&\parbox{10cm}
	{\begin{align}
	 \myvec{\frac{\partial f_1}{\partial x_1} & \frac{\partial f_1}{\partial x_2} &\dots & \frac{\partial f_1}{\partial x_n} \\
	 \frac{\partial f_2}{\partial x_1} & \frac{\partial f_2}{\partial x_2} &\dots & \frac{\partial f_2}{\partial x_n} \\
    \vdots &\vdots& \ddots & \vdots\\
    \frac{\partial f_m}{\partial x_1} & \frac{\partial f_m}{\partial x_2} &\dots & \frac{\partial f_m}{\partial x_n}} \label{eq:2}
	\end{align}}\\ 
	&\\
	\hline
	\multirow{3}{*}{Inner product} 
	&\\
	& The inner product of $\vec{x}$ and $\vec{y}$ is given by\\
    &\parbox{10cm}
	{\begin{align}
	\langle \vec{x},\vec{y} \rangle =\vec{x}^T\vec{y}=\vec{y}^T\vec{x}  
	\end{align}}\\ 
	&\\
    \hline
    \caption{Definition and Properties used}
    \label{table:1}
\end{longtable}
\newpage
\section{\textbf{Solution}}
\renewcommand{\thetable}{2}
\begin{longtable}{|l|l|}
	\hline
	\multirow{3}{*}{Given}
	&\\
    &\parbox{10cm}
	{\begin{align}
	F(\vec{x},\vec{y})=\langle \vec{A}\vec{x},\vec{y} \rangle 
	\end{align}}\\
    \hline
	\multirow{3}{*}{using inner product definition}
	& \\
	&\parbox{10cm}
	{\begin{align}
	F(\vec{x},\vec{y})=(\vec{A}\vec{x})^T\vec{y}=\vec{x}^T\vec{A}^T\vec{y}\\
	F(\vec{x},\vec{y})=\vec{y}^T\vec{A}\vec{x}
	\end{align}}\\
	&\\
	\hline
	\multirow{3}{*}{Derivative of F}
	&\\
	&using \eqref{eq:2}, We can write that\\
	&\parbox{10cm}
	{\begin{align}
	DF(\vec{x},\vec{y})=\myvec{\frac{\partial F}{\partial x}&\frac{\partial F}{\partial y}}
	=\myvec{\vec{y}^T\vec{A} & \vec{x}^T\vec{A}^T} \label{eq:main}
	\end{align}}\\
	&\\
	\hline
	\multirow{3}{*}{If $\vec{x} \neq 0$,then $DF(\vec{x},0 ) \neq 0$}
	&\\
	&using \eqref{eq:main},\\
	&\parbox{10cm}
	{\begin{align}
	DF(\vec{x},0)=\myvec{0 & \vec{x}^T\vec{A}^T} \\
	\text{From $\eqref{eq:1}$,we know that} \nonumber \\
	\vec{A}\vec{x} \neq 0\\
	\implies \vec{x}^T\vec{A}^T \neq 0
	\end{align}}\\
	&So, We can say that \\
	&\parbox{10cm}
	{\begin{align}
    DF(\vec{x},0)\neq 0 \label{option1}
	\end{align}}\\
	&\\
    \hline
    \multirow{3}{*}{If $\vec{y} \neq 0$,then $DF(0,\vec{y} ) \neq 0$}
	&\\
	&using \eqref{eq:main},\\
	&\parbox{10cm}
	{\begin{align}
	DF(0,\vec{y})=\myvec{\vec{y}^T\vec{A}  & 0} \\
	\text{From $\eqref{eq:1.1}$,we know that} \nonumber \\
	\vec{A}^T\vec{y} \neq 0\\
	\implies \vec{y}^T\vec{A} \neq 0
	\end{align}}\\
	&So, We can say that \\
	&\parbox{10cm}
	{\begin{align}
    DF(0,\vec{y}) \neq 0 \label{option2}
	\end{align}}\\
	&\\
	 \hline
    \multirow{3}{*}{If $(\vec{x},\vec{y}) \neq 0$,then $DF(\vec{x},\vec{y} ) \neq 0$}
	&\\
	&using \eqref{eq:main},\\
	&\parbox{10cm}
	{\begin{align}
	DF(\vec{x},\vec{y}) =\myvec{\vec{y}^T\vec{A} & \vec{x}^T\vec{A}^T}\\
	\text{As $(\vec{x},\vec{y}) \neq 0,DF(\vec{x},\vec{y}) = 0 $ iff $\vec{A}$=0 }\nonumber \\
	\text{From $\eqref{eq:1}$,we know that} \nonumber \\
	\vec{A} \neq 0
	\end{align}}\\
	&So, We can say that \\
	&\parbox{10cm}
	{\begin{align}
    DF(\vec{x},\vec{y}) \neq 0\label{option3}
	\end{align}}\\
	&\\
    \hline
    \multirow{3}{*}{If $\vec{x} = 0$ or $\vec{y}=0$,then $DF(\vec{x},\vec{y})= 0$}
	&\\
	&From \eqref{option2},\\
	&\parbox{10cm}
	{\begin{align}
    DF(0,\vec{y}) \neq 0  
	\end{align}}\\
	&From \eqref{option1},\\
	&\parbox{10cm}
	{\begin{align}
    DF(\vec{x},0)\neq 0 
	\end{align}}\\
	&So, if $\vec{x} = 0$ or $\vec{y}=0$,\\
	&\parbox{10cm}
	{\begin{align}
    DF(\vec{x},\vec{y}) \neq 0
	\end{align}}\\
    &\\
    \hline
	\multirow{3}{*}{Conclusion} & \\
	& From above,we can say that options 1),2),3) are correct.\\
    &\\
	\hline
	\caption{Finding derivative of linear transformation}
    \label{table:2}
\end{longtable}
\end{document}
