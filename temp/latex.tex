\documentclass[journal,12pt]{IEEEtran}
\usepackage{longtable}
\usepackage{setspace}
\usepackage{gensymb}
\singlespacing
\usepackage[cmex10]{amsmath}
\newcommand\myemptypage{
	\null
	\thispagestyle{empty}
	\addtocounter{page}{-1}
	\newpage
}
\usepackage{amsthm}
\usepackage{mdframed}
\usepackage{mathrsfs}
\usepackage{txfonts}
\usepackage{stfloats}
\usepackage{bm}
\usepackage{cite}
\usepackage{cases}
\usepackage{subfig}

\usepackage{longtable}
\usepackage{multirow}

\usepackage{enumitem}
\usepackage{mathtools}
\usepackage{steinmetz}
\usepackage{tikz}
\usepackage{circuitikz}
\usepackage{verbatim}
\usepackage{tfrupee}
\usepackage[breaklinks=true]{hyperref}
\usepackage{graphicx}
\usepackage{tkz-euclide}

\usetikzlibrary{calc,math}
\usepackage{listings}
    \usepackage{color}                                            %%
    \usepackage{array}                                            %%
    \usepackage{longtable}                                        %%
    \usepackage{calc}                                             %%
    \usepackage{multirow}                                         %%
    \usepackage{hhline}                                           %%
    \usepackage{ifthen}                                           %%
    \usepackage{lscape}     
\usepackage{multicol}
\usepackage{chngcntr}

\DeclareMathOperator*{\Res}{Res}

\renewcommand\thesection{\arabic{section}}
\renewcommand\thesubsection{\thesection.\arabic{subsection}}
\renewcommand\thesubsubsection{\thesubsection.\arabic{subsubsection}}

\renewcommand\thesectiondis{\arabic{section}}
\renewcommand\thesubsectiondis{\thesectiondis.\arabic{subsection}}
\renewcommand\thesubsubsectiondis{\thesubsectiondis.\arabic{subsubsection}}


\hyphenation{op-tical net-works semi-conduc-tor}
\def\inputGnumericTable{}                                 %%

\lstset{
%language=C,
frame=single, 
breaklines=true,
columns=fullflexible
}
\begin{document}
\onecolumn

\newtheorem{theorem}{Theorem}[section]
\newtheorem{problem}{Problem}
\newtheorem{proposition}{Proposition}[section]
\newtheorem{lemma}{Lemma}[section]
\newtheorem{corollary}[theorem]{Corollary}
\newtheorem{example}{Example}[section]
\newtheorem{definition}[problem]{Definition}

\newcommand{\BEQA}{\begin{eqnarray}}
\newcommand{\EEQA}{\end{eqnarray}}
\newcommand{\define}{\stackrel{\triangle}{=}}
\bibliographystyle{IEEEtran}
\raggedbottom
\setlength{\parindent}{0pt}
\providecommand{\mbf}{\mathbf}
\providecommand{\pr}[1]{\ensuremath{\Pr\left(#1\right)}}
\providecommand{\qfunc}[1]{\ensuremath{Q\left(#1\right)}}
\providecommand{\sbrak}[1]{\ensuremath{{}\left[#1\right]}}
\providecommand{\lsbrak}[1]{\ensuremath{{}\left[#1\right.}}
\providecommand{\rsbrak}[1]{\ensuremath{{}\left.#1\right]}}
\providecommand{\brak}[1]{\ensuremath{\left(#1\right)}}
\providecommand{\lbrak}[1]{\ensuremath{\left(#1\right.}}
\providecommand{\rbrak}[1]{\ensuremath{\left.#1\right)}}
\providecommand{\cbrak}[1]{\ensuremath{\left\{#1\right\}}}
\providecommand{\lcbrak}[1]{\ensuremath{\left\{#1\right.}}
\providecommand{\rcbrak}[1]{\ensuremath{\left.#1\right\}}}
\theoremstyle{remark}
\newtheorem{rem}{Remark}
\newcommand{\sgn}{\mathop{\mathrm{sgn}}}
\providecommand{\abs}[1]{\left\vert#1\right\vert}
\providecommand{\res}[1]{\Res\displaylimits_{#1}} 
\providecommand{\norm}[1]{\left\lVert#1\right\rVert}
%\providecommand{\norm}[1]{\lVert#1\rVert}
\providecommand{\mtx}[1]{\mathbf{#1}}
\providecommand{\mean}[1]{E\left[ #1 \right]}
\providecommand{\fourier}{\overset{\mathcal{F}}{ \rightleftharpoons}}
%\providecommand{\hilbert}{\overset{\mathcal{H}}{ \rightleftharpoons}}
\providecommand{\system}{\overset{\mathcal{H}}{ \longleftrightarrow}}
	%\newcommand{\solution}[2]{\textbf{Solution:}{#1}}
\newcommand{\solution}{\noindent \textbf{Solution: }}
\newcommand{\cosec}{\,\text{cosec}\,}
\providecommand{\dec}[2]{\ensuremath{\overset{#1}{\underset{#2}{\gtrless}}}}
\newcommand{\myvec}[1]{\ensuremath{\begin{pmatrix}#1\end{pmatrix}}}
\newcommand{\mydet}[1]{\ensuremath{\begin{vmatrix}#1\end{vmatrix}}}
\numberwithin{equation}{subsection}
\makeatletter
\@addtoreset{figure}{problem}
\makeatother
\let\StandardTheFigure\thefigure
\let\vec\mathbf
\renewcommand{\thefigure}{\theproblem}
\def\putbox#1#2#3{\makebox[0in][l]{\makebox[#1][l]{}\raisebox{\baselineskip}[0in][0in]{\raisebox{#2}[0in][0in]{#3}}}}
     \def\rightbox#1{\makebox[0in][r]{#1}}
     \def\centbox#1{\makebox[0in]{#1}}
     \def\topbox#1{\raisebox{-\baselineskip}[0in][0in]{#1}}
     \def\midbox#1{\raisebox{-0.5\baselineskip}[0in][0in]{#1}}
\vspace{3cm}
\title{Assignment 17}
\author{M Pavan Manesh - EE20MTECH14017}
\maketitle
\bigskip
\renewcommand{\thefigure}{\theenumi}
\renewcommand{\thetable}{\theenumi}
%
Download the latex-tikz codes from 
%
\begin{lstlisting}
https://github.com/pavanmanesh/EE5609/tree/master/Assignment17
\end{lstlisting}
\section{\textbf{Problem}}
Consider the quadratic forms on $\mathbb{R}^2$
\begin{align}
    Q_{1}(x,y)=xy,Q_{2}(x,y)=x^2+2xy+y^2,Q_{3}(x,y)=x^2+3xy+2y^2 \nonumber
\end{align}
Choose the correct statements from below
\begin{enumerate}
\item $Q_{1}$ and $Q_{2}$ are equivalent
\item $Q_{1}$ and $Q_{3}$ are equivalent
\item $Q_{2}$ and $Q_{3}$ are equivalent
\item All are equivalent
\end{enumerate}
\section{\textbf{Definitions}}
\renewcommand{\thetable}{1}
\begin{longtable}{|l|l|}
	\hline
	\multirow{3}{*}{Matrix representation} 
	& \\
	& The Matrix representation of quadratic forms\\
	&\parbox{10cm}
	{\begin{align}
	Q(x,y) = ax^2+2bxy+cy^2=\myvec{x&y}\myvec{a&b\\b&c}\myvec{x\\y} =\vec{X}^{T}\vec{A}\vec{X}\label{eq:1}
	\end{align}}\\
	&The symmetric matrix of the quadratic form is\\
	&\parbox{10cm}
	{\begin{align}
	\vec{A}=\myvec{a&b\\b&c}
	\end{align}}\\ 
	&\\
	\hline
	\multirow{3}{*}{Equivalent condition} 
	&\\
	& Two quadratic forms $\vec{X}^{T}\vec{A}\vec{X}$ and $\vec{Y}^{T}\vec{B}\vec{Y}$ are called equivalent if their matrices,\\
	& A and B are congruent.\\
	&\\
	& Two real quadratic forms are equivalent over the real field iff they have\\
	& the same rank and the same index.\\
	&\\
    \hline
    \multirow{3}{*}{Rank} 
	&\\
	& The rank of a quadratic form is the rank of its associated matrix.\\
	&\\
	\hline
    \multirow{3}{*}{Index} 
	&\\
	& The index of the quadratic form is equal to the number of positive eigen\\ &values of the matrix of quadratic form.\\
	&\\
    \hline
    \caption{Definitions and results used}
    \label{table:1}
\end{longtable}
\renewcommand{\thetable}{2}
\section{\textbf{Solution}}
\begin{table*}[ht]
\begin{center}
\centering
\begin{tabular}{|l|l|l|l|} 
\hline
    & $Q_{1}(x,y)$ & $Q_{2}(x,y)$ & $Q_{3}(x,y)$  \\ 
\hline
Matrix Representation   & $\vec{A}_1=\myvec{0&\frac{1}{2}\\\frac{1}{2}&0}$                                     & $\vec{A}_2=\myvec{1&1\\1&1}$                                                         & $\vec{A}_3=\myvec{1&\frac{3}{2}\\\frac{3}{2}&2}$                                     \\ 
\hline
Finding Rank       
& \begin{tabular}[c]{@{}l@{}}
\\
$\myvec{0&\frac{1}{2}\\\frac{1}{2}&0}\xleftrightarrow[R_2\leftarrow R_1]{R_1\leftarrow R_2}
\myvec{\frac{1}{2}&0\\0&\frac{1}{2}}$\\
\\
$\xleftrightarrow[R_2\leftarrow 2R_2]{R_1\leftarrow 2R_1}
\myvec{1&0\\0&1}$
\end{tabular}  
& \begin{tabular}[c]{@{}l@{}}
    $\myvec{1&1\\1&1}\xleftrightarrow[]{R_2\leftarrow R_2-R_1}\myvec{1&1\\0&0}$\\
\end{tabular}
& \begin{tabular}[c]{@{}l@{}}
\\
$\myvec{1&\frac{3}{2}\\\frac{3}{2}&2}
\xleftrightarrow[]{R_2\leftarrow R_2-\frac{3}{2} R_1}
\myvec{1&\frac{3}{2}\\0&-\frac{1}{4}}$\\
\\
$\xleftrightarrow[R_2\leftarrow -4R_2]{R_1\leftarrow R_1+6R_2}
\myvec{1&0\\0&1}$
\end{tabular}  \\
\hline
Rank      & $\text{rank}(\vec{A}_1)=2$                                                           & $\text{rank}(\vec{A}_2)=1$                                                           & $\text{rank}(\vec{A}_3)=2$                                                          \\ 
\hline
Finding Eigen values      
& \begin{tabular}[c]{@{}l@{}}
$\mydet{\Vec{A}_1-\lambda\Vec{I}}=0$\\
$\implies\mydet{-\lambda&\frac{1}{2}\\\frac{1}{2}&-\lambda}=0$\\
$\implies\brak{\lambda-\frac{1}{2}}\brak{\lambda+\frac{1}{2}}=0$\\
$\implies \lambda_1=\frac{1}{2},\lambda_2=-\frac{1}{2}$
\\
\end{tabular} 
& \begin{tabular}[c]{@{}l@{}}
$\mydet{\Vec{A}_2-\lambda\Vec{I}}=0$\\
$\implies\mydet{1-\lambda&1\\1&1-\lambda}=0$
\\$\implies\brak{\lambda}\brak{\lambda-2}=0$
\\$\implies \lambda_1=0,\lambda_2=2$
\\
\end{tabular}
& \begin{tabular}[c]{@{}l@{}}
$\mydet{\Vec{A}_3-\lambda\Vec{I}}=0$\\
$\implies\mydet{1-\lambda&\frac{3}{2}\\\frac{3}{2}&2-\lambda}=0$\\
$\implies\brak{\lambda-\frac{\sqrt{10}+3}{2}}\brak{\lambda+\frac{\sqrt{10}-3}{2}}=0$\\
$\implies \lambda_1=\frac{3+\sqrt{10}}{2},\lambda_2=\frac{3-\sqrt{10}}{2}$\\
\end{tabular}      
\\ 
\hline
Index
& \text{Index of $\vec{A}_1$}  =1
& \text{Index of $\vec{A}_2$}=2
& \text{Index of $\vec{A}_3$}=1                                                  \\ 
\hline
Conclusion & \multicolumn{3}{l|}{}From above,we can say $Q_{1}$ and $Q_{3}$ are equivalent.\\
\hline
\end{tabular}
\caption{Finding which quadratic forms are equivalent}
    \label{table:2}
\end{center}
\end{table*}
\end{document}
