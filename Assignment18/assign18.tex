\documentclass[journal,12pt]{IEEEtran}
\usepackage{longtable}
\usepackage{setspace}
\usepackage{gensymb}
\singlespacing
\usepackage[cmex10]{amsmath}
\newcommand\myemptypage{
	\null
	\thispagestyle{empty}
	\addtocounter{page}{-1}
	\newpage
}
\usepackage{amsthm}
\usepackage{mdframed}
\usepackage{mathrsfs}
\usepackage{txfonts}
\usepackage{stfloats}
\usepackage{bm}
\usepackage{cite}
\usepackage{cases}
\usepackage{subfig}

\usepackage{longtable}
\usepackage{multirow}

\usepackage{enumitem}
\usepackage{mathtools}
\usepackage{steinmetz}
\usepackage{tikz}
\usepackage{circuitikz}
\usepackage{verbatim}
\usepackage{tfrupee}
\usepackage[breaklinks=true]{hyperref}
\usepackage{graphicx}
\usepackage{tkz-euclide}

\usetikzlibrary{calc,math}
\usepackage{listings}
    \usepackage{color}                                            %%
    \usepackage{array}                                            %%
    \usepackage{longtable}                                        %%
    \usepackage{calc}                                             %%
    \usepackage{multirow}                                         %%
    \usepackage{hhline}                                           %%
    \usepackage{ifthen}                                           %%
    \usepackage{lscape}     
\usepackage{multicol}
\usepackage{chngcntr}

\DeclareMathOperator*{\Res}{Res}

\renewcommand\thesection{\arabic{section}}
\renewcommand\thesubsection{\thesection.\arabic{subsection}}
\renewcommand\thesubsubsection{\thesubsection.\arabic{subsubsection}}

\renewcommand\thesectiondis{\arabic{section}}
\renewcommand\thesubsectiondis{\thesectiondis.\arabic{subsection}}
\renewcommand\thesubsubsectiondis{\thesubsectiondis.\arabic{subsubsection}}


\hyphenation{op-tical net-works semi-conduc-tor}
\def\inputGnumericTable{}                                 %%

\lstset{
%language=C,
frame=single, 
breaklines=true,
columns=fullflexible
}
\begin{document}
\onecolumn

\newtheorem{theorem}{Theorem}[section]
\newtheorem{problem}{Problem}
\newtheorem{proposition}{Proposition}[section]
\newtheorem{lemma}{Lemma}[section]
\newtheorem{corollary}[theorem]{Corollary}
\newtheorem{example}{Example}[section]
\newtheorem{definition}[problem]{Definition}

\newcommand{\BEQA}{\begin{eqnarray}}
\newcommand{\EEQA}{\end{eqnarray}}
\newcommand{\define}{\stackrel{\triangle}{=}}
\bibliographystyle{IEEEtran}
\raggedbottom
\setlength{\parindent}{0pt}
\providecommand{\mbf}{\mathbf}
\providecommand{\pr}[1]{\ensuremath{\Pr\left(#1\right)}}
\providecommand{\qfunc}[1]{\ensuremath{Q\left(#1\right)}}
\providecommand{\sbrak}[1]{\ensuremath{{}\left[#1\right]}}
\providecommand{\lsbrak}[1]{\ensuremath{{}\left[#1\right.}}
\providecommand{\rsbrak}[1]{\ensuremath{{}\left.#1\right]}}
\providecommand{\brak}[1]{\ensuremath{\left(#1\right)}}
\providecommand{\lbrak}[1]{\ensuremath{\left(#1\right.}}
\providecommand{\rbrak}[1]{\ensuremath{\left.#1\right)}}
\providecommand{\cbrak}[1]{\ensuremath{\left\{#1\right\}}}
\providecommand{\lcbrak}[1]{\ensuremath{\left\{#1\right.}}
\providecommand{\rcbrak}[1]{\ensuremath{\left.#1\right\}}}
\theoremstyle{remark}
\newtheorem{rem}{Remark}
\newcommand{\sgn}{\mathop{\mathrm{sgn}}}
\providecommand{\abs}[1]{\left\vert#1\right\vert}
\providecommand{\res}[1]{\Res\displaylimits_{#1}} 
\providecommand{\norm}[1]{\left\lVert#1\right\rVert}
%\providecommand{\norm}[1]{\lVert#1\rVert}
\providecommand{\mtx}[1]{\mathbf{#1}}
\providecommand{\mean}[1]{E\left[ #1 \right]}
\providecommand{\fourier}{\overset{\mathcal{F}}{ \rightleftharpoons}}
%\providecommand{\hilbert}{\overset{\mathcal{H}}{ \rightleftharpoons}}
\providecommand{\system}{\overset{\mathcal{H}}{ \longleftrightarrow}}
	%\newcommand{\solution}[2]{\textbf{Solution:}{#1}}
\newcommand{\solution}{\noindent \textbf{Solution: }}
\newcommand{\cosec}{\,\text{cosec}\,}
\providecommand{\dec}[2]{\ensuremath{\overset{#1}{\underset{#2}{\gtrless}}}}
\newcommand{\myvec}[1]{\ensuremath{\begin{pmatrix}#1\end{pmatrix}}}
\newcommand{\mydet}[1]{\ensuremath{\begin{vmatrix}#1\end{vmatrix}}}
\numberwithin{equation}{subsection}
\makeatletter
\@addtoreset{figure}{problem}
\makeatother
\let\StandardTheFigure\thefigure
\let\vec\mathbf
\renewcommand{\thefigure}{\theproblem}
\def\putbox#1#2#3{\makebox[0in][l]{\makebox[#1][l]{}\raisebox{\baselineskip}[0in][0in]{\raisebox{#2}[0in][0in]{#3}}}}
     \def\rightbox#1{\makebox[0in][r]{#1}}
     \def\centbox#1{\makebox[0in]{#1}}
     \def\topbox#1{\raisebox{-\baselineskip}[0in][0in]{#1}}
     \def\midbox#1{\raisebox{-0.5\baselineskip}[0in][0in]{#1}}
\vspace{3cm}
\title{Assignment 18}
\author{M Pavan Manesh - EE20MTECH14017}
\maketitle
\bigskip
\renewcommand{\thefigure}{\theenumi}
\renewcommand{\thetable}{\theenumi}
%
Download the latex-tikz codes from 
%
\begin{lstlisting}
https://github.com/pavanmanesh/EE5609/tree/master/Assignment18
\end{lstlisting}
\section{\textbf{Problem}}
Let $\vec{A}$ be a $\brak{6 \times 6}$ matrix over $\mathbb{R}$ with characteristic polynomial =$\brak{x-3}^2\brak{x-2}^4$ and the minimum polynomial =$\brak{x-3}\brak{x-2}^2$.Then jordan canonical form of $\vec{A}$ can be
\begin{enumerate}
\item \myvec{3&0&0&0&0&0\\0&3&0&0&0&0\\0&0&2&1&0&0\\0&0&0&2&1&0\\0&0&0&0&2&1\\0&0&0&0&0&2}
\item \myvec{3&0&0&0&0&0\\0&3&0&0&0&0\\0&0&2&1&0&0\\0&0&0&2&0&0\\0&0&0&0&2&0\\0&0&0&0&0&2}
\item \myvec{3&0&0&0&0&0\\0&3&0&0&0&0\\0&0&2&1&0&0\\0&0&0&2&0&0\\0&0&0&0&2&1\\0&0&0&0&0&2}
\item \myvec{3&1&0&0&0&0\\0&3&0&0&0&0\\0&0&2&1&0&0\\0&0&0&2&0&0\\0&0&0&0&2&1\\0&0&0&0&0&2}
\end{enumerate}
\newpage
\section{\textbf{Definitions}}
\renewcommand{\thetable}{1}
\begin{longtable}{|l|l|}
	\hline
	\multirow{3}{*}{Jordan canonical form} 
	&\\
	& If $\vec{A}$ is a matrix of order n$\times$n ,then the Jordan canonical form\\
	&of $\vec{A}$  is a matrix of order n$\times$n expressed as \\
	&\parbox{10cm}
	{\begin{align}
	\vec{J} = \myvec{\vec{J_1} & & \\
    & \ddots & \\
    & &
    \vec{J_{k}}} \label{eq:1}
	\end{align}}\\ 
	&\text{where} $\vec{J_1},...,\vec{J_k}$ are the Jordan blocks. \\
	&\\
	\hline
	\multirow{3}{*}{Algebraic multiplicity $A_M$} 
	& \\
	& Algebraic multiplicity of characteristic value $\lambda$ in the characteristic \\ 
	& polynomial determines the size of Jordan block for that eigen value\\
	&\parbox{10cm}
	{\begin{align}
	A_M= \text{Size of Jordan block for that $\lambda$ } \label{eq:1}
	\end{align}}\\ 
	&\\
	\hline
	\multirow{3}{*}{Geometric multiplicity $G_M$} 
	&\\
	& Geometric multiplicity determines the number of Jordan sub-blocks \\
	&in a Jordan block for $\lambda$\\
	&\\
    \hline
    	\multirow{3}{*}{Minimal Polynomial}
	&\\
	&The multiplicity of $\lambda$ in the minimal polynomial determines the\\
	&  size of the largest sub-block.\\
	&\\
    \hline
    \caption{Definition and Properties used}
    \label{table:1}
\end{longtable}
\section{\textbf{Solution}}
\renewcommand{\thetable}{2}
\begin{longtable}{|l|l|}
	\hline
	\multirow{3}{*}{Characteristic polynomial}
	&\\
    &\parbox{10cm}
	{\begin{align}
	p\brak{x}=\brak{x-3}^2\brak{x-2}^4
	\end{align}}\\
    \hline
	\multirow{3}{*}{Algebraic Multiplicity}
	& \\
	&\parbox{10cm}
	{\begin{align}
	\text{For }\lambda=3, A_M=2\\ \text{For }\lambda=2, A_M=4 
	\end{align}}\\
	&\\
	\hline
	\multirow{3}{*}{Minimal polynomial}
	&\\
    &\parbox{10cm}
	{\begin{align}
	m\brak{x}=\brak{x-3}\brak{x-2}^2
	\end{align}}\\
	&\\
	\hline
	\multirow{3}{*}{Finding Jordan blocks for $\lambda_{1}$=3}
	&\\
	&For $\lambda_{1}$=3,We can write from table\ref{table:1} that\\
	&\parbox{10cm}
	{\begin{align}
	\text{The highest order of Jordan block}=1 \nonumber\\
	\text{Size of Jordan block}=A_M=2 \nonumber
	\end{align}}\\
	&The Jordan blocks for $\lambda_{1}$=3\\
	&\parbox{10cm}
	{\begin{align}
	\vec{J_1}=\myvec{3},
	\vec{J_2}=\myvec{3}
	\end{align}}\\
	&\\
	\hline
	\multirow{3}{*}{Finding Jordan blocks for $\lambda_{1}$=2}
	&\\
	&For $\lambda_{1}$=2,We can write from table\ref{table:1} that\\
	&\parbox{10cm}
	{\begin{align}
	\text{The highest order of Jordan block}=2 \nonumber\\
	\text{Size of Jordan block}=A_M=4 \nonumber
	\end{align}}\\
	&The Jordan blocks for $\lambda_{1}$=3\\
	&\parbox{10cm}
	{\begin{align}
	\vec{J_3}=\myvec{2&1\\0&2},
	\vec{J_4}=\myvec{2&1\\0&2}\\
	\text{or} \nonumber\\ 
	\vec{J_3}=\myvec{2&1\\0&2},
	\vec{J_4}=\myvec{2},\vec{J_5}=\myvec{2}
	\end{align}}\\
	&\\
    \hline
	\multirow{3}{*}{Jordan canonical form} 
	&\\
	& Jordan canonical form of $\vec{A}$ is \\
	&\parbox{10cm}
	{\begin{align}
	\vec{J} = \myvec{\vec{J_1} & & \\
	& \vec{J_2} & \\
    && \vec{J_3} & \\
    &&& \vec{J_{4}} } \text{or}
    \myvec{\vec{J_1} & & \\
	& \vec{J_2} & \\
    && \vec{J_3} & \\
    &&& \vec{J_{4}} & \\
    &&&& \vec{J_{5} } }
	\end{align}}\\ 
	&\parbox{10cm}
	{\begin{align}
	\myvec{3&0&0&0&0&0\\0&3&0&0&0&0\\0&0&2&1&0&0\\0&0&0&2&0&0\\0&0&0&0&2&1\\0&0&0&0&0&2}\text{or}
	\myvec{3&0&0&0&0&0\\0&3&0&0&0&0\\0&0&2&1&0&0\\0&0&0&2&0&0\\0&0&0&0&2&0\\0&0&0&0&0&2}
	\end{align}}\\ 
	& \\
    \hline
	\multirow{3}{*}{Conclusion} & \\
	& From above,we can say that options 2) and 3) are correct.\\
    &\\
	\hline
	\caption{Finding Jordan canonical form}
    \label{table:2}
\end{longtable}
\end{document}
