\documentclass[journal,12pt]{IEEEtran}
\usepackage{longtable}
\usepackage{setspace}
\usepackage{gensymb}
\singlespacing
\usepackage[cmex10]{amsmath}
\newcommand\myemptypage{
	\null
	\thispagestyle{empty}
	\addtocounter{page}{-1}
	\newpage
}
\usepackage{amsthm}
\usepackage{mdframed}
\usepackage{mathrsfs}
\usepackage{txfonts}
\usepackage{stfloats}
\usepackage{bm}
\usepackage{cite}
\usepackage{cases}
\usepackage{subfig}

\usepackage{longtable}
\usepackage{multirow}

\usepackage{enumitem}
\usepackage{mathtools}
\usepackage{steinmetz}
\usepackage{tikz}
\usepackage{circuitikz}
\usepackage{verbatim}
\usepackage{tfrupee}
\usepackage[breaklinks=true]{hyperref}
\usepackage{graphicx}
\usepackage{tkz-euclide}

\usetikzlibrary{calc,math}
\usepackage{listings}
    \usepackage{color}                                            %%
    \usepackage{array}                                            %%
    \usepackage{longtable}                                        %%
    \usepackage{calc}                                             %%
    \usepackage{multirow}                                         %%
    \usepackage{hhline}                                           %%
    \usepackage{ifthen}                                           %%
    \usepackage{lscape}     
\usepackage{multicol}
\usepackage{chngcntr}

\DeclareMathOperator*{\Res}{Res}

\renewcommand\thesection{\arabic{section}}
\renewcommand\thesubsection{\thesection.\arabic{subsection}}
\renewcommand\thesubsubsection{\thesubsection.\arabic{subsubsection}}

\renewcommand\thesectiondis{\arabic{section}}
\renewcommand\thesubsectiondis{\thesectiondis.\arabic{subsection}}
\renewcommand\thesubsubsectiondis{\thesubsectiondis.\arabic{subsubsection}}


\hyphenation{op-tical net-works semi-conduc-tor}
\def\inputGnumericTable{}                                 %%

\lstset{
%language=C,
frame=single, 
breaklines=true,
columns=fullflexible
}
\begin{document}
\onecolumn

\newtheorem{theorem}{Theorem}[section]
\newtheorem{problem}{Problem}
\newtheorem{proposition}{Proposition}[section]
\newtheorem{lemma}{Lemma}[section]
\newtheorem{corollary}[theorem]{Corollary}
\newtheorem{example}{Example}[section]
\newtheorem{definition}[problem]{Definition}

\newcommand{\BEQA}{\begin{eqnarray}}
\newcommand{\EEQA}{\end{eqnarray}}
\newcommand{\define}{\stackrel{\triangle}{=}}
\bibliographystyle{IEEEtran}
\raggedbottom
\setlength{\parindent}{0pt}
\providecommand{\mbf}{\mathbf}
\providecommand{\pr}[1]{\ensuremath{\Pr\left(#1\right)}}
\providecommand{\qfunc}[1]{\ensuremath{Q\left(#1\right)}}
\providecommand{\sbrak}[1]{\ensuremath{{}\left[#1\right]}}
\providecommand{\lsbrak}[1]{\ensuremath{{}\left[#1\right.}}
\providecommand{\rsbrak}[1]{\ensuremath{{}\left.#1\right]}}
\providecommand{\brak}[1]{\ensuremath{\left(#1\right)}}
\providecommand{\lbrak}[1]{\ensuremath{\left(#1\right.}}
\providecommand{\rbrak}[1]{\ensuremath{\left.#1\right)}}
\providecommand{\cbrak}[1]{\ensuremath{\left\{#1\right\}}}
\providecommand{\lcbrak}[1]{\ensuremath{\left\{#1\right.}}
\providecommand{\rcbrak}[1]{\ensuremath{\left.#1\right\}}}
\theoremstyle{remark}
\newtheorem{rem}{Remark}
\newcommand{\sgn}{\mathop{\mathrm{sgn}}}
\providecommand{\abs}[1]{\left\vert#1\right\vert}
\providecommand{\res}[1]{\Res\displaylimits_{#1}} 
\providecommand{\norm}[1]{\left\lVert#1\right\rVert}
%\providecommand{\norm}[1]{\lVert#1\rVert}
\providecommand{\mtx}[1]{\mathbf{#1}}
\providecommand{\mean}[1]{E\left[ #1 \right]}
\providecommand{\fourier}{\overset{\mathcal{F}}{ \rightleftharpoons}}
%\providecommand{\hilbert}{\overset{\mathcal{H}}{ \rightleftharpoons}}
\providecommand{\system}{\overset{\mathcal{H}}{ \longleftrightarrow}}
	%\newcommand{\solution}[2]{\textbf{Solution:}{#1}}
\newcommand{\solution}{\noindent \textbf{Solution: }}
\newcommand{\cosec}{\,\text{cosec}\,}
\providecommand{\dec}[2]{\ensuremath{\overset{#1}{\underset{#2}{\gtrless}}}}
\newcommand{\myvec}[1]{\ensuremath{\begin{pmatrix}#1\end{pmatrix}}}
\newcommand{\mydet}[1]{\ensuremath{\begin{vmatrix}#1\end{vmatrix}}}
\numberwithin{equation}{subsection}
\makeatletter
\@addtoreset{figure}{problem}
\makeatother
\let\StandardTheFigure\thefigure
\let\vec\mathbf
\renewcommand{\thefigure}{\theproblem}
\def\putbox#1#2#3{\makebox[0in][l]{\makebox[#1][l]{}\raisebox{\baselineskip}[0in][0in]{\raisebox{#2}[0in][0in]{#3}}}}
     \def\rightbox#1{\makebox[0in][r]{#1}}
     \def\centbox#1{\makebox[0in]{#1}}
     \def\topbox#1{\raisebox{-\baselineskip}[0in][0in]{#1}}
     \def\midbox#1{\raisebox{-0.5\baselineskip}[0in][0in]{#1}}
\vspace{3cm}
\title{Assignment 15}
\author{M Pavan Manesh - EE20MTECH14017}
\maketitle
\bigskip
\renewcommand{\thefigure}{\theenumi}
\renewcommand{\thetable}{\theenumi}
%
Download the latex-tikz codes from 
%
\begin{lstlisting}
https://github.com/pavanmanesh/EE5609/tree/master/Assignment15
\end{lstlisting}
\section{\textbf{Problem}}
 Let n be a positive integer, and let $V$ be the space of polynomials over $\mathbb{R}$ which have degree at most n (throw in
the 0-polynomial). Let $\vec{D}$ be the differentiation operator on $V$. What is the minimal polynomial for $\vec{D}$?
\section{\textbf{Definitions}}
\renewcommand{\thetable}{1}
\begin{longtable}{|l|l|}
	\hline
	\multirow{3}{*}{Characteristic Polynomial} 
	& \\
	& For an $n\times n$ matrix $\vec{A}$, characteristic polynomial is defined by,\\
	&\\
	& $\qquad\qquad\qquad p\brak{x}=\mydet{x\Vec{I}-\Vec{A}}$\\
	&\\
	\hline
	\multirow{3}{*}{Minimal Polynomial} 
	&\\
	& Minimal polynomial $m\brak{x}$ is the smallest factor of characteristic polynomial\\
	& $p\brak{x}$ such that,\\
	&\\
	& $\qquad \qquad \qquad m\brak{\vec{A}}=0$\\
	& \\
	& Every root of characteristic polynomial should be the root of minimal\\
	& polynomial\\
	&\\
    \hline
    \caption{Definitions and theorem used}
    \label{table:1}
\end{longtable}
\newpage
\section{\textbf{Solution}}
\renewcommand{\thetable}{2}
\begin{longtable}{|l|l|}
	\hline
	\multirow{3}{*}{Given} & \\
	& $V$ is the space of polynomials over $\mathbb{R}$ which have degree at most n.\\
    & \\
    \hline
	\multirow{3}{*}{Matrix Representation}
	& \\
	& The basis for the space $V$ is \\
	&\parbox{10cm}
	{\begin{align}
	\mathcal{B}=\cbrak{1,x,x^2,\dots,x^n}
	\end{align}}\\
	& Given that $\vec{D}$ is the differentiation operator.So,\\
	&\parbox{10cm}
	{\begin{align}
	\vec{D}(1)=0\\
	\vec{D}(x)=1\\
	\vdots \nonumber \\
	\vec{D}(x^n)=nx^{n-1}
	\end{align}}\\
	&The vectors of differentiation operator with respect to basis $\mathcal{B}$ \\
	&\parbox{10cm}
	{\begin{align}
	[\vec{D}\brak{1}]_{\mathcal{B}}=\myvec{0\\0\\\vdots\\0}_{(n+1)\times 1},
	[\vec{D}\brak{x}]_{\mathcal{B}}=\myvec{1\\0\\\vdots\\0}_{(n+1)\times 1} \dots
	[\vec{D}\brak{x^n}]_{\mathcal{B}}=\myvec{0\\\vdots\\n\\0}_{(n+1)\times 1}
	\end{align}}\nonumber\\
	&The matrix representation can be written as:\\
	&\parbox{10cm}
	{\begin{align}
	\vec{A}=
	\myvec{0&1&0&\dots&0\\
	0&0&2&\dots&0\\
	\vdots & \vdots & \vdots & \dots & \vdots\\
	0&0&0&\dots&n\\
	0&0&0&\dots&0}
	\end{align}}\\
	&\\
	\hline
	\multirow{3}{*}{Characteristic polynomial} 
	& \\
    &\parbox{10cm}
	{\begin{align}
	p\brak{x}=\mydet{x\Vec{I}-\Vec{A}}=
	\mydet{x&-1&0&\dots&0\\
	0&x&-2&\dots&0\\
	\vdots & \vdots & \vdots & \dots & \vdots\\
	0&0&0&\dots&-n\\
	0&0&0&\dots&x}
	\end{align}}\\
	&It is equal to the product of diagonal entries.\\
	&\parbox{10cm}
	{\begin{align}
	p\brak{x}=x^{n+1}
	\end{align}}\\
	&\\
	\hline
	\multirow{3}{*}{Minimal Polynomial} & \\
	& The minimal polynomial of $\vec{A}$ can be any of $x,x^2,\dots,x^{n+1}$ such that,\\
	&\parbox{10cm}
	{\begin{align}
	m\brak{\vec{A}}=0
	\end{align}}\\
	\hline
	\multirow{3}{*}{Explanation} & \\
	&Let $P(n)$: Minimum polynomial of $\vec{D}$=$x^{n+1}$ i.e $\vec{A}^{n+1}=0$ \\
	& For n=1\\
	&\parbox{10cm}
	{\begin{align}
	\vec{A}=\myvec{0&1\\0&0}\\
	\vec{A^2}=\myvec{0&0\\0&0}
	\end{align}}\\
    &So,$P(1)$ is true.\\
    &Assume $P(k)$ holds for $1 \le k \le n$.\\
	&\parbox{10cm}
	{\begin{align}
	\vec{A}_k=
	\myvec{0&1&0&\dots&0\\
	0&0&2&\dots&0\\
	\vdots & \vdots & \vdots & \dots & \vdots\\
	0&0&0&\dots&k\\
	0&0&0&\dots&0}_{(k+1 \times k+1)}
	\implies \vec{A}_k^{k+1}=\vec{0} \label{1}
	\end{align}}\\
	& We need to show that $P(k+1)$ is true.\\
	&\parbox{10cm}
	{\begin{align}
	\vec{A}_{k+1}=
	\myvec{0&1&0&\dots&0&0\\
	0&0&2&\dots&0&0\\
	\vdots & \vdots & \vdots & \dots & \vdots& \vdots\\
	0&0&0&\dots&k&0\\
	0&0&0&\dots&0&k+1\\
	0&0&0&\dots&0&0}_{(k+2 \times k+2)}
	\end{align}}\\
	& Expressing in terms of block matrices \\
	&\parbox{10cm}
	{\begin{align}
	\vec{A}_{k+1}=\myvec{\vec{A}_k&\vec{x}\\\vec{0}_{1\times k+1}&0},
    \vec{x}=\myvec{0\\0\\\vdots\\0\\k+1}_{k+1 \times 1}
	\end{align}}\\
	&\\
	\hline
	\multirow{3}{*}{Finding $\vec{A}_{k+1}^{k+2}$} & \\
	&\parbox{10cm}
	{\begin{align}
	\vec{A}_{k+1}^2=\myvec{\vec{A}_k&\vec{x}\\\vec{0}&0}
	\myvec{\vec{A}_k&\vec{x}\\\vec{0}&0}=
	\myvec{\vec{A}_k^2&\vec{A}_k\vec{x}\\0&0}\\
	\vec{A}_{k+1}^3=\myvec{\vec{A}_k^2&\vec{A}_k\vec{x}\\0&0}
	\myvec{\vec{A}_k&\vec{x}\\\vec{0}&0}
	=\myvec{\vec{A}_k^3&\vec{A}_k^2\vec{x}\\0&0}\\
	\vec{A}_{k+1}^{k+2}=\myvec{\vec{A}_k^{k+2}&\vec{A}_k^{k+1}\vec{x}\\0&0}\\
	\text{From \eqref{1},We know that $\vec{A}_k^{k+1}=\vec{0}$} \nonumber\\
	\implies \vec{A}_{k+1}^{k+2}=\vec{0}
	\end{align}}\\
	&So,$P(k+1)$ is true.\\
	&\\
	&\\
    \hline
	\multirow{3}{*}{Conclusion} & \\
	& From above,by using the principle of induction we can say that\\ &the minimal polynomial is\\
	&\parbox{10cm}
	{\begin{align}
	x^{n+1}
	\end{align}}\\
	& \\
	\hline
	\caption{Finding minimal polynomial}
    \label{table:2}
\end{longtable}
\end{document}
